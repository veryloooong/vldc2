\section[Chương 2]{Chương 2: Vật dẫn}

\subsection[Câu 1]{Câu 1: Trình bày
\begin{itemize}
    \item Điều kiện cân bằng tĩnh điện của một vật dẫn mang điện.
    \item Các tính chất của vật dẫn tích điện cân bằng (có chứng minh)
    \item Nêu ứng dụng về tính chất của vật dẫn tích điện cân bằng
\end{itemize}
}

Điều kiện cân bằng tĩnh điện của vật dẫn mang điện:

\begin{enumerate}
  \item Vector cường độ điện trường tại mọi điểm trong vật dẫn bằng 0
  \begin{equation*}
    \vec{E}_{tr} = 0
  \end{equation*}
  \item Thành phần tiếp tuyến của vector cường độ điện trường tại mọi điểm trên mặt vật dẫn bằng 0
  \begin{equation*}
    \vec{E}_t = 0, \vec{E}_n = \vec{E}
  \end{equation*}
\end{enumerate}

Các tính chất của vật dẫn tích điện cân bằng:

\begin{enumerate}
  \item Vật dẫn là vật đẳng thế
  \begin{itemize}
    \item Xét 2 điểm $M, N$ trên vật dẫn:
    \begin{equation*}
      V_M - V_N = \int_{M}^{N} \vec{E}d\vec{s} = \int_{M}^{N} E_sds
    \end{equation*}
    \item Bên trong vật dẫn thì $E_s = 0$ nên $V$ tại mọi điểm bên trong bằng nhau
    \item Bên ngoài vật dẫn thì $E_s = E_t = 0$ nên $V$ tại mọi điểm ngoài bằng nhau
    \item Vì $V$ có tính liên tục nên $V$ tại mọi điểm như nhau
  \end{itemize}
  \item Khi vật dẫn ở trạng thái cân bằng tĩnh điện, điện tích chỉ phân bố ở bên ngoài vật dẫn. Bên trong điện tích bằng 0
  \begin{itemize}
    \item Xét 1 mặt kín $S$ bất kỳ bên trong vật dẫn. Theo O-G có
    \begin{equation*}
      \sum q_i = \oint_S \vec{D}d\vec{s}
    \end{equation*}
    \item Bên trong vật dẫn có 
    \begin{equation*}
      \vec{D} = \epsilon_0\epsilon\vec{E} = \vec{0} \Rightarrow \sum q_i = 0
    \end{equation*}
    Vậy bên trong vật dẫn điện tích bằng 0
  \end{itemize}
  \item Sự phân bố điện tích trên mặt vật dẫn chỉ phụ thuộc vào hình dạng mặt đó
  \begin{itemize}
    \item Điện tích tập trung ở những chỗ có mũi nhọn
    \item Ở những chỗ lõm điện tích ít, gần như bằng 0
  \end{itemize}
\end{enumerate}

Ứng dụng về tính chất của vật dẫn tích điện cân bằng:

\begin{itemize}
  \item Máy phát điện Vandegraf
  \item Giải phóng điện tích
  \item Cột thu lôi
\end{itemize}

\subsection[Câu 2]{Câu 2: Định nghĩa hiện tượng điện hưởng. Thế nào là hai phần tử tương ứng? Phát biểu định lý các phần tử tương ứng. Thế nào là hiện tượng điện hưởng một phần và điện hưởng toàn phần?}

Đặt vật dẫn trung hòa trong điện trường ngoài $\vec{E}_0$ thì hai phía của vật dẫn xuất hiện các điện tích trái dấu gọi là điện tích cảm ứng $\Rightarrow$ hiện tượng cộng hưởng.\\

Phần tử tương ứng:
\begin{itemize}
  \item Xét vật dẫn trung hòa $BC$ đặt trong điện trường ngoài của quả cầu $A$ tích điện dương
  \item Xét tập hợp đường cảm ứng điện tựa trên chu vi của phần tử diện tích $\Delta S$ trên $A$
  \item Giả sử tập hợp đường cảm ứng điện tới tận cùng chu vi của phần tử diện tích $\Delta S'$ trên $BC$. Có $\Delta S$ và $\Delta S'$ là các phần tử tương ứng
\end{itemize}

Định lý các phần tử tương ứng:
\begin{itemize}
  \item Tưởng tượng mặt kín $S$ là ống các đường cảm ứng điện và hai mặt lấy trên $\Delta S$ và $\Delta S'$
  \item Theo O-G có 
  \begin{equation*}
    \Phi = \int D_ndS = \sum q_i = \Delta q + \Delta q'
  \end{equation*}
  Có $D = 0 \Rightarrow \Delta q = - \Delta q'$ 
  \item Điện tích cảm ứng trên các phần tử tương ứng có độ lớn bằng nhau và trái dấu.
\end{itemize}

Điện hưởng 1 phần và toàn phần:
\begin{itemize}
  \item Điện hưởng 1 phần: Chỉ 1 phần số đường cảm ứng điện gặp vật bị điện hưởng: $|q'| < |q|$
  \item Điện hưởng toàn phần: Khi vật dẫn bao bọc vật mang điện, toàn bộ số đương cảm ứng gặp vật bị điện hưởng: $|q'| = |q|$
\end{itemize}

\subsection[Câu 3]{Câu 3: Định nghĩa tụ điện. Thiết lập biểu thức điện dung của tụ điện phẳng và tụ điện cầu.}

Tụ điện là hệ 2 vật dẫn A và B sao cho B bao bọc hoàn toàn A. Khi đó 2 vật ở trạng thái điện hưởng toàn phần.

Công thức chung: $Q = CU$

\begin{itemize}
  \item Điện dung của tụ phẳng: $C = \frac{\epsilon_0\epsilon S}{d}$
  \item Điện dung của tụ cầu: $C = \frac{4\pi\epsilon_0\epsilon R_1R_2}{R_2-R_1}$
  \item Điện dung của tụ trụ: $C = \frac{2\pi\epsilon_0\epsilon l}{\ln{R_2 / R_1}}$
\end{itemize}
 
\subsection[Câu 4]{Câu 4: Trình bày năng lượng tương tác của hệ điện tích điểm, năng lượng của vật dẫn mang điện và năng lượng của tụ điện.}

Năng lượng tương tác của hệ tích điểm:

\begin{equation*}
  W = \frac{1}{2} \sum q_iV_i
\end{equation*}

Năng lượng tương tác của một vật dẫn mang điện:

\begin{equation*}
  W = \frac{1}{2} \int Vdq = \frac{1}{2} qV = \frac{1}{2} CV^2 = \frac{q^2}{2C}
\end{equation*}

Năng lượng của tụ điện:

\begin{equation*}
  W = \frac{1}{2} \left( q_1V_1 + q_2V_2 \right) = \frac{1}{2} qU = \frac{1}{2} CU^2 = \frac{q^2}{2C}
\end{equation*}

\subsection[Câu 5]{Câu 5: Tính năng lượng điện trường của tụ điện phẳng tích điện từ đó suy ra công thức mật độ năng lượng điện trường và năng lượng của một điện trường bất kỳ.}

Năng lượng điện trường của tụ điện phẳng:

\begin{equation*}
  W = \frac{1}{2} CU^2 = \frac{1}{2} \frac{\epsilon_0\epsilon S}{d} (Ed)^2 = \frac{1}{2}\epsilon_0\epsilon E^2 \Delta V
\end{equation*}

Mật độ năng lượng điện trường:

\begin{equation*}
  w_e = \frac{W}{\Delta V} = \frac{1}{2} \epsilon_0\epsilon E^2 = \frac{1}{2}DE
\end{equation*}

Năng lượng điện trường của một điện trường bất kỳ:

\begin{equation*}
  W = \int w_e dV = \frac{1}{2} \int \epsilon_0\epsilon E^2dV
\end{equation*}

