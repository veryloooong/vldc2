\section[Chương 8]{Chương 8: Dao động điện từ}

\subsection[Câu 1]{Câu 1: Thiết lập biểu thức trong dao động điện từ điều hòa.}

Xét mạch $LC$. Trong quá trình dao động năng lượng không đổi: $W_e + W_m = \text{const}$

\begin{align*}
  {}\quad & \frac{1}{2} \frac{q^2}{C} + \frac{1}{2} LI^2 = \text{const} \\
  \Rightarrow\quad & \frac{q}{C} \frac{dq}{dt} + LI \frac{dI}{dt} = 0 \\
  \Rightarrow\quad & \frac{q}{C} + L\frac{dI}{dt} = 0 \\
  \Rightarrow\quad & \frac{d^2I}{dt^2} + \frac{1}{LC} I = 0
\end{align*}

Đặt $\omega_0^2 = \frac{1}{LC}$, ta có $\frac{d^2I}{dt^2} + \omega_0^2 I = 0$

\begin{equation*}
  I = I_0 \cos\left( \omega_0 t + \varphi \right)
\end{equation*}

\subsection[Câu 2]{Câu 2: Thiết lập biểu thức trong dao động điện từ tắt dần.}

Xét mạch $RLC$. Do tỏa nhiệt Joule -- Lenz biên độ giảm dần. 

\begin{itemize}
  \item Trong thời gian $dt$, năng lượng mạch giảm một lượng $-dW$
  \item Nhiệt lượng tỏa ra là $RI^2dt$
\end{itemize}

\begin{align*}
  {}\quad & -dW = RI^2dt \\
  \Rightarrow\quad & -d\left( \frac{1}{2} \frac{q^2}{C} + \frac{1}{2} LI^2 \right) = RI^2dt \\
  \Rightarrow\quad & \frac{q}{C} \frac{dq}{dt} + LI \frac{dI}{dt} = -RI^2 \\
  \Rightarrow\quad & \frac{q}{C} + L\frac{dI}{dt} + RI = 0 \\ 
  \Rightarrow\quad & \frac{1}{C} \frac{dq}{dt} + L\frac{d^2I}{dt^2} + R\frac{dI}{dt} = 0 \\
  \Rightarrow\quad & \frac{d^2I}{dt^2} + \frac{R}{L} \frac{dI}{dt} + \frac{1}{LC} I = 0
\end{align*}

Đặt $\omega_0^2 = \frac{1}{LC}$ và $\beta = \frac{R}{2L}$. Khi đó có $\frac{d^2I}{dt^2} + 2\beta\frac{dI}{dt} + \omega_0^2 I = 0$

\begin{equation*}
  I = I_0 e^{-\beta t} \cos\left( \omega t + \varphi \right)
\end{equation*}

với $\omega = \sqrt{\omega_0^2 - \beta^2}$

\subsection[Câu 3]{Câu 3: Thiết lập biểu thức trong dao động điện từ cưỡng bức.}

Xét thế điện động tuần hoàn gắn với mạch $RLC$

\begin{itemize}
  \item Thế điện động là hàm tuần hoàn theo thời gian: $E = E_0 \sin \Omega t$
  \item Trong thời gian $dt$, nguồn cung cấp năng lượng $EIdt$
  \item Năng lượng này là phần tăng năng lượng điện từ của mạch cùng với nhiệt lượng Joule -- Lenz
\end{itemize}

\begin{align*}
  {}\quad & EIdt = d\left( \frac{1}{2}\frac{q^2}{C} + \frac{1}{2} LI^2 \right) + RI^2dt \\
  \Rightarrow\quad & E_0\sin\Omega t \cdot I = \frac{q}{C} \frac{dq}{dt} + LI\frac{dI}{dt} + RI^2 \\
  \Rightarrow\quad & E_0\sin\Omega t = \frac{q}{C} + L\frac{dI}{dt} + RI \\
  \Rightarrow\quad & \frac{d^2I}{dt^2} + \frac{R}{L} \frac{dI}{dt} + \frac{1}{LC} I = \frac{E_0 \Omega}{L} \cos \Omega t \\
\end{align*}

Đặt $\omega_0^2 = \frac{1}{LC}$ và $\beta = \frac{R}{2L}$. Khi đó $\frac{d^2I}{dt^2} + 2\beta\frac{dI}{dt} + \omega_0^2 I = \frac{E_0 \Omega}{L} \cos \Omega t$

\begin{equation*}
  I = I_0 e^{-\beta t} \cos (\omega t + \varphi) + I_0 \cos \left( \Omega t + \Phi \right)
\end{equation*}

với $I_0 = \frac{E_0}{\sqrt{R^2 + \left( \Omega L - \frac{1}{\Omega C} \right)^2}}$