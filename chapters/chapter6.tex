\section[Chương 6]{Chương 6: Vật liệu từ}

\subsection[Câu 1]{Câu 1: 
\begin{enumerate}
  \item Thế nào là chất nghịch từ, thuận từ, sắt từ?
  \item Trình bày về vecto từ độ $\vec{J}$.
\end{enumerate}
}

\subsubsection{Chất nghịch từ, thuận từ, sắt từ}

Mọi chất đặt trong từ trường đều bị từ hóa. Khi đó chúng trở thành mang từ tính và sinh ra một từ trường phụ $\vec{B}'$. Từ trường tổng hợp $\vec{B}$ trong chất bị từ hóa là 

\begin{equation*}
  \vec{B} = \vec{B}_0 + \vec{B}'
\end{equation*}

\begin{itemize}
  \item Chất nghịch từ: $\vec{B}'$ ngược chiều $\vec{B}_0 \Rightarrow |\vec{B}| < |\vec{B}_0|$ 
  \item Chất thuận từ: $\vec{B}'$ thuận chiều $\vec{B}_0 \Rightarrow |\vec{B}| > |\vec{B}_0|$
  \item Chất sắt từ: $\vec{B}'$ thuận chiều $\vec{B}_0 \Rightarrow |\vec{B}| \gg |\vec{B}_0|$
\end{itemize}

\subsubsection{Từ độ $\vec{J}$}

Vector từ độ là momen từ của một đơn vị thể tích của khối vật liệu từ.

Gọi $\sum \vec{P}_{mi}$ là tổng các momen từ nguyên tử có trong một thể tích $\Delta V$ của vật liệu. Nếu khối vật liệu bị từ hóa đồng đều thì vector từ độ $\vec{J}$ là 

\begin{equation*}
  \vec{J} = \frac{\sum \vec{P}_{mi}}{\Delta V}
\end{equation*}

$J = |\vec{J}|$ là từ độ của vật liệu từ.

Thực nghiệm cho thấy $\vec{J}$ tỉ lệ thuận với từ trường ngoài $\vec{B}_0$.

\begin{equation*}
  \begin{cases}
    \vec{J} = \frac{\chi_m}{\mu_0} \vec{B}_0 \\
    \vec{B}_0 = \mu_0 \vec{H}
  \end{cases}
  \Rightarrow \vec{J} = \chi_m \vec{H}
\end{equation*}

$\chi_m$ là hệ số tỷ lệ phụ thuộc vào bản chất của vật liệu từ và gọi là độ từ hóa của vật liệu từ.

\subsection[Câu 2]{Câu 2: Xây dựng công thức tính cảm ứng từ tổng hợp trong chất thuận từ đồng chất và đẳng hướng.}

\begin{itemize}
  \item Xét khối thuận từ đồng nhất hình trụ dài vô hạn, tiết diện thẳng $S$
  \item Trụ đặt trong từ trường đều $\vec{B}_0$ có phương song song với đường sinh
  \item Giả sử dưới tác dụng của từ trường ngoài, các momen nguyên tử $\vec{P}_m$ nằm dọc theo hướng $\vec{B}_0$
\end{itemize}

Xét các dòng điện nguyên tử trong 1 tiết diện thẳng của trụ:

\begin{itemize}
  \item Bên trong tiết diện thì các dòng điện nguyên tử ngược chiều và triệt tiêu lẫn nhau
  \item Chỉ có các dòng điện nằm dọc trên chu vi của tiết diện là cùng chiều và tạo 1 dòng điện tròn chạy quanh chu vi của tiết diện
\end{itemize}

Vậy nếu xét cả hình trụ thì tất cả các dòng điện nguyên tử trong các tiết diện thẳng sẽ tương đương với 1 dòng điện duy nhất chạy quanh mặt ngoài của trụ như 1 ống dây điện thẳng dài vô hạn.

\begin{itemize}
  \item $n_0$ là số dòng điện tròn trên 1m hình trụ
  \item $i$ là cường độ dòng điện
  \item Cảm ứng phụ sinh ra là $B' = \mu_0 n_0 i$
\end{itemize}

Mặt khác có

\begin{gather*}
  J = \frac{n_0 i S}{S} = n_0 i \\
  \Rightarrow B' = \mu_0 J
\end{gather*}

Vì $\vec{B}'$ thuận chiều $\vec{J} \Rightarrow \vec{B}' = \mu_0 \vec{J}$

Từ trường tổng hợp là 

\begin{equation*}
  \begin{cases}
    \vec{B} = \vec{B}_0 + \mu_0 \vec{J} \\
    \vec{J} = \frac{\chi_m}{\mu_0} \vec{B}_0
  \end{cases}
  \Rightarrow \vec{B} = (1 + \chi_m) \vec{B}_0 = \mu \vec{B}_0 = \mu_0 \mu \vec{H}
\end{equation*}

Lý luận trên cũng đúng với chất nghịch từ.

\begin{itemize}
  \item Thuận từ: $\chi_m > 0 \Rightarrow \mu > 1$
  \item Nghịch từ: $\chi_m < 0 \Rightarrow \mu < 1$
\end{itemize}