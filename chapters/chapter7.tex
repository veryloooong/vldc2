\section[Chương 7]{Chương 7: Trường điện từ}

\subsection[Câu 1]{Câu 1: Phát biểu luận điểm 1 của Maxwell. Phân biệt điện trường tĩnh và điện trường xoáy về nguồn gốc phát sinh và tính chất cơ bản. Thiết lập phương trình Maxwell -- Faraday dạng tích phân và vi phân.}

Luận điểm thứ nhất của Maxwell: Bất kỳ từ trường nào biến đổi theo thời gian cũng sinh ra một điện trường xoáy.

Phân biệt điện trường tĩnh và điện trường xoáy:

\begin{table}[H]
\centering
\begin{tabular}{|p{6cm}|p{6cm}|}
\hline
Điện trường tĩnh & Điện trường xoáy \\ \hline\hline
Điện trường tĩnh sinh ra bởi điện tích đứng yên & Điện trường xoáy sinh ra bởi từ trường biến thiên theo thời gian \\ \hline
Là điện trường hở & Là điện trường kín \\ \hline
\end{tabular}
\end{table}

Phương trình Maxwell -- Faraday dạng tích phân:

\begin{equation*}
  \oint_C \vec{E}d\vec{l} = -\frac{d}{dt} \left( \int_C \vec{B}d\vec{S} \right)
\end{equation*}

Lưu số của vector cường độ điện trường dọc theo đường cong kín bất kỳ bằng về trị số nhưng trái dấu với tốc độ biến thiên theo thời gian của từ thông gửi qua diện tích giới hạn bởi đường cong đó.

Phương trình Maxwell -- Faraday dạng vi phân:

\begin{equation*}
  \text{rot}\,\vec{E} = -\frac{\partial\vec{B}}{\partial t}
\end{equation*}

\subsection[Câu 2]{Câu 2: Phát biểu luận điểm 2 của Maxwell. Khái niệm dòng điện dịch. So sánh
dòng điện dịch và dòng điện dẫn. Thiết lập phương trình Maxwell -- Ampere dạng tích phân và vi phân.}

Luận điểm thứ hai của Maxwell: Bất kì một điện trường nào biến đổi theo thời gian cũng sinh ra một từ trường.

Dòng điện dịch là dòng điện tương đương với điện trường biến đổi theo thời gian về phương diện sinh ra từ trường.

Phân biệt dòng điện dẫn và dòng điện dịch:

\begin{table}[H]
\centering
\begin{tabular}{|p{6cm}p{6cm}|}
\hline
\multicolumn{1}{|p{6cm}|}{ Dòng điện dẫn  }& Dòng điện dịch \\ \hline\hline
\multicolumn{1}{|p{6cm}|}{ Là dòng chuyển dời có hướng của các hạt mang điện  }& Là $\vec{E}$ biến thiên theo $t$ \\ \hline
\multicolumn{1}{|p{6cm}|}{ Gây ra tỏa nhiệt theo Joule -- Lenz  }& Không gây ra tỏa nhiệt theo Joule -- Lenz \\ \hline
\multicolumn{2}{|c|}{Đều gây ra từ trường $\vec{B}$} \\ \hline
\end{tabular}
\end{table}

Phương trình Maxwell -- Ampere dạng tích phân:

\begin{equation*}
  \oint_C \vec{H}d\vec{l} = \int_S \left( \vec{J}_{\text{dẫn}} + \frac{\partial \vec{D}}{\partial t} \right) d\vec{S}
\end{equation*}

Lưu số của vector cường độ từ trường dọc theo đường cong kín bất kỳ bằng cường độ dòng điện toàn phần chạy qua diện tích giới hạn bởi đường cong đó.

Phương trình Maxwell -- Ampere dạng vi phân:

\begin{equation*}
  \text{rot}\,\vec{H} = \vec{J}_{\text{dẫn}} + \frac{\partial \vec{D}}{\partial t}
\end{equation*}